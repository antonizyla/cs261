\documentclass{article}
% set up the page formatting
\usepackage[a4paper, portrait, margin=2.5cm]{geometry}
\usepackage{multicol}
\usepackage{fancyhdr}
\usepackage{xcolor}
\usepackage{graphicx}
% editable bits
\title{CS261 Group 29 Planning and Design}
\author{Add your names here}
\date{January 2025}
\fancyfoot[L]{Planning and Design}
\fancyfoot[R]{\thepage}

\begin{document}
\maketitle
\tableofcontents

\section{Given Requirements for document}
You are also required to submit a Planning and Design Document. This should address the design of your software solution together with a detailed plan as to how (and when) this design is to be implemented. There are a number of things to consider in the design of software systems, including:

\begin{itemize}
    \item Extensibility – how you can extend your solution given increasing demands;
    \item Robustness – how your solution will tolerate unpredictable or invalid input;
    \item Reliability – how the system will perform under everyday conditions;
    \item Correctness – how accurately your solution meets the requirements of the customer;
    \item Compatibility and portability – how easy your proposed system is to install and execute;
    \item Modularity and reuse – how well your system is divided into independent components and whether you have reused existing code;
    \item Security – whether your system can withstand hostile acts and influences;
    \item Fault-tolerance – whether your system can withstand and recover from component failure.
\end{itemize}

This list is not exhaustive and there will be other concerns which your group will want to address. You will also want to adopt trusted design patterns or design methodologies to provide a template for the actual design of your system. No one method will be favoured (by the assessors) over another, but you will be graded on the process of selecting an appropriate methodology and your use of it.
Each group will have to submit these two short reports on tabula on Monday 3rd February 2025. Each member of the team should submit these reports along with a team contribution form for the first half of the project (see below). 

\section{Team Planning}
\subsection{Time Management}
In our meetings we hae discussed how we are going to manage our time and have agreed to the following deadlines:

\begin{itemize}
    \item Requirement Analysis: 22/01/2025
    \item Planning and Design: 29/01/2025
    \item Back-End Development and Testing: 26/02/2025
    \item Front-End Development and Testing: 26/02/2025
    \item Dragons' Den Video: 1/03/2025
    \item Final Report: 10/03/2025
\end{itemize}

Below you can see a Gantt chart of our planned work schedule.

\begin{figure}[h!]
    \centering
    \includegraphics[width=\textwidth]{ganttchart.png}
    \caption{Gantt Chart of Planned Work Schedule}
    \label{fig:gantt_chart}
\end{figure}


\subsection{Risk Assessment and Management}
We have identified the following risks and have agreed on the following mitigation strategies:
\textbf{Risk 1:} Technology Limitations
\begin{itemize}
    \item Risk description: Team/Team members may be unfamiliar to some tool, libraries, or frameworks, which may cause delays or reduced performance. 
    \item Risk Level: \textcolor{Green}{Tolerable}
    \item Risk Likelihood: \textcolor{BurntOrange}{Moderate}
    \item Mitigation Strategy: Assign tasks to team members based on their expertise in relevant technologies while ensuring everyone is involved in meaningful roles to maintain prductivity and foster teamwork.
\end{itemize}


\textbf{Risk 2:} Rollback Challenges
\begin{itemize}
    \item Risk description: Lack of a version control system could prevent from rolling back to the software's last stable state in case of errors
    \item Risk Level: \textcolor{Red}{Catastrophic}
    \item Risk Likelihood: \textcolor{Green}{Low}
    \item Mitigation Strategy: Utilise github to always maintain a stable version of the software and updating it when being sure that the changes will not affect its usability. 
\end{itemize}


\textbf{Risk 3:} Testing Risks	
\begin{itemize}
    \item Risk description: Insufficient testing may reduce confidence in the software
    \item Risk Level: \textcolor{BurntOrange}{Serious}
    \item Risk Likelihood: \textcolor{BurntOrange}{Moderate}
    \item Mitigation Strategy: Unit tests will be designed to test the software to make sure that it is working properly.
\end{itemize}

\textbf{Risk 4:} Time Management
\begin{itemize}
    \item Risk description: Underestimating  task duration or improper prioritization might result in delayed work.
    \item Risk Level: \textcolor{BurntOrange}{Serious}
    \item Risk Likelihood: \textcolor{Green}{Low}
    \item Mitigation Strategy: The team is meeting in regular intervals to ensure work efficiency and mitigate time related risks.
\end{itemize}

\textbf{Risk 5:} Requirement Misalignment 
\begin{itemize}
    \item Risk description: During the development of the software, the end product might not be the same as the one describe in the delivrables due to unforeseen circumstances
    \item Risk Level: \textcolor{Red}{Catastrophic}
    \item Risk Likelihood: \textcolor{Green}{Low}
    \item Mitigation Strategy: Ensure constant internal communication between the team.
\end{itemize}

\textbf{Risk 6:} Organisational Risks
\begin{itemize}
    \item Risk description: Uneven distribution of workload or misscommunication may lead to an uncomplete project and delayed work. 
    \item Risk Level: \textcolor{BurntOrange}{Serious}
    \item Risk Likelihood: \textcolor{Green}{Low}
    \item Mitigation Strategy: The team is meeting in regular intervals to ensure work efficiency and mitigate time related risks.
\end{itemize}

\textbf{Risk 7:} Team Member MIA
\begin{itemize}
    \item Risk description: Team member is not able to complete their amount of work due to unforeseen circumstances, thus delaying work. 
    \item Risk Level: \textcolor{BurntOrange}{Serious}
    \item Risk Likelihood: \textcolor{BurntOrange}{Moderate}
    \item Mitigation Strategy: Team analyses the remaining work from missing member and prioritises and reallocates tasks based on the analysis. 
\end{itemize}

\section{Front-End}
%Discussion of using Python and Pygame/similar libraries to create the user interface (including parsing user input) and graphics

\section{Back-End}
%Discussion of the design of the simulation
For the back-end, we have decided to implement it in Python due to the whole group being familiar with the language, and to make 
interfacing between the front-end and back-end simple. We plan to use objects to simulate the junction configurations and calculate 
the junction efficiency metrics and overall scores, with the structure of classes being as follows:
\begin{figure}[H]
    \centering
    \includegraphics[width=0.5\linewidth]{JunctionSimulationClassDiagram.drawio.pdf}
    \caption{Junction Simulation Class Diagram}
    \label{class diagram}
\end{figure}
The Junction class contains Lane objects representing each lane of a road entering the junction, grouped together in arrays based on 
the direction they’re arriving at the junction from. Two functions will be called by the front-end to set up the simulation: 
setFlowRates to set the flow rates from each direction to the other directions, passed in the form of a 2D array, and 
setJunctionConfiguration to set the specific settings of the junction (e.g. is there a left turn lane, how many lanes there are on 
each incoming road).

Each Lane object contains a queue of Vehicle objects, representing the vehicles in that lane waiting to transit the junction. Each Lane will calculate its own maximum wait time (maxWait), average wait time (avgWait) and maximum queue length (maxLength), which will be compared with the same values of the other Lanes in its direction by the Junction object, in order to get the three values for the direction. 


\section{Deployment}
%This is where we discuss containerisation
\end{document}
