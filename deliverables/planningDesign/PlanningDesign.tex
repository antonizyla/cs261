\documentclass{article}
% set up the page formatting
\usepackage[a4paper, portrait, margin=2.5cm]{geometry}
\usepackage{multicol}
\usepackage{fancyhdr}
% editable bits
\title{CS261 Group 29 Planning and Design}
\author{Add your names here}
\date{January 2025}
\fancyfoot[L]{Planning and Design}
\fancyfoot[R]{\thepage}

\begin{document}
\maketitle
\tableofcontents

\section{Given Requirements for document}
You are also required to submit a Planning and Design Document. This should address the design of your software solution together with a detailed plan as to how (and when) this design is to be implemented. There are a number of things to consider in the design of software systems, including:

\begin{itemize}
    \item Extensibility – how you can extend your solution given increasing demands;
    \item Robustness – how your solution will tolerate unpredictable or invalid input;
    \item Reliability – how the system will perform under everyday conditions;
    \item Correctness – how accurately your solution meets the requirements of the customer;
    \item Compatibility and portability – how easy your proposed system is to install and execute;
    \item Modularity and reuse – how well your system is divided into independent components and whether you have reused existing code;
    \item Security – whether your system can withstand hostile acts and influences;
    \item Fault-tolerance – whether your system can withstand and recover from component failure.
\end{itemize}

This list is not exhaustive and there will be other concerns which your group will want to address. You will also want to adopt trusted design patterns or design methodologies to provide a template for the actual design of your system. No one method will be favoured (by the assessors) over another, but you will be graded on the process of selecting an appropriate methodology and your use of it.
Each group will have to submit these two short reports on tabula on Monday 3rd February 2025. Each member of the team should submit these reports along with a team contribution form for the first half of the project (see below). 

\section{Front-End}
%Discussion of using Python and Pygame/similar libraries to create the user interface (including parsing user input) and graphics

\section{Back-End}
%Discussion of the design of the simulation

\section{Deployment}
%This is where we discuss containerisation
\end{document}
