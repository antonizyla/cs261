\documentclass{article}
% set up the page formatting
\usepackage[a4paper, portrait, margin=2.5cm]{geometry}
\usepackage{multicol}
\usepackage{fancyhdr}
% editable bits
\title{CS261 Group 29 Requirement Analysis}
\author{Add your names here, Antoni Zyla}
\date{January 2025}
\fancyfoot[L]{Requirement Analysis}
\fancyfoot[R]{\thepage}

\begin{document}
\maketitle

\section{Given Requirements for document}
Each team is required to provide a Requirements Analysis Report. This will 
document your teams understanding of the specification. As well as addressing 
the specification itself, the report should also document any decisions that 
have been made in terms of group management (i.e. how you have organized 
yourselves to meet the goals of this project, how often you will meet, how you 
will address decision making etc.) and provide a list of requirements that 
your team have decided will meet the customer's needs.
A requirements analysis report should provide an outline of the features that 
the final solution will provide. Remember that requirements need to be detailed, 
justified and verifiable. Consider your validation process carefully, along 
with why you've chosen to include certain features. This part of your first 
deliverable must not exceed four sides of A4. Any appendices or other attachments 
will be removed.

\section{Introduction}
% Justifies the need for the system and outlines what it will do.

\section{User Requirements Design} \label{requirements_user}
% Describes the services provided for the users. Written in natural language with diagrams.
\subsection{Requirement 1.C}
The user must be able to input the rate of traffic flow from 
each direction to each other direction via text fields (one per traffic flow) 
in the User Interface (UI).
\subsection{Requirement 2.C}
The user must be able to adjust the following settings for each 
road entering the junction, affecting the results of the simulation: how many 
lanes there are (1 to 5), whether there is a left-turn lane, bus lane, cycle 
lane (all mutually exclusive, the latter two requiring separate traffic flow rates) 
or none of those three, whether there is a pedestrian crossing (requires the duration 
of time the crossing is active and the number of crossing requests per hour), 
and the traffic light sequencing priority (0 to 4, 0 meaning no priority, 4 
meaning highest priority).
\subsection{Requirement 3.C}
After running the simulation, for each junction configuration, 
the user must be provided with the following three junction efficiency metrics 
per road entering the junction, as well as an overall score generated from the 
three metrics for the whole configuration: average wait time, maximum wait time, 
and maximum queue length.

\section{System Architecture}
% Presents a high-level overview of the system, showing distribution of functions across system modules.

\section{System Requirements Specification}
% Describes functional and non-functional requirements.
\subsection{Functional Requirements}
\subsubsection{Requirement 1.D}
\begin{itemize}
  \item The system must accept input from the text fields, parse the input to 
  determine if each traffic flow is a valid number, and, if all traffic flows 
  are valid, allow the simulation to be ran.
  \item Priority: must
  \item Verification: running the simulation with one set of valid traffic 
  flow rates in the text fields, running it again with a different set in 
  the same fields and verifying the junction efficiency metrics have changed, 
  and then trying to run the simulation with an invalid value in at least one 
  text field (e.g. “TEST” for Eastbound Traffic Flow Exiting West)
  \item Traceability: Input Parameters section in the specification
\end{itemize}

\subsubsection{Requirement 2.D}
\begin{itemize}
  \item For each junction configuration, the system must take into account 
  its specific settings, adjusting the calculations performed by the simulation 
  for that specific configuration
  \item Priority: must
  \item Verification: run a simulation with default parameters (2 lanes per 
  road entering the junction, equal priority on all lights, and all other 
  settings set to off/No) as a reference, and then run a simulation per setting 
  with that setting adjusted, showing the results of each simulation is different 
  from the reference.
  \item Traceability: Configurable Parameters section in the specification
\end{itemize}

\subsubsection{Requirement 3.D}
\begin{itemize}
  \item When the simulation is ran, for each junction configuration, the system 
  must take the configuration and the input traffic flow rates (shared across all 
  configurations) and calculate the average wait time, maximum wait time and 
  maximum queue length per direction entering the junction, as well as combine 
  these metrics to calculate an overall score for the whole configuration
  \item Priority: must
  \item Verification: run the simulation once with two different configurations 
  or twice with a single configuration and different traffic flows, and verify 
  that there is a difference between the metrics and overall score of the 
  configurations/simulation runs
  \item Traceability: Output section in the specification
\end{itemize}

\subsubsection{Requirement 4.D}
\begin{itemize}
  \item The system must allow for comparison of one or more sets of input 
  parameters against the metrics defined in section \ref{requirements_user}, 
  the system would then state which set of parameters produces the best traffic 
  flow based on each of the metrics, for example A,B are both intersections, 
  A has a higher overall throughput but very long queues on one of the directions, 
  whereas B has a lower overall throughput but distributes the queues more evenly, 
  this would allow the user to make an informed choice.
  \item Priority: Must 
\end{itemize}

\subsection{Non-functional Requirements}
\subsubsection{Requirement 1}
\begin{itemize}
  \item The system will show graphical representation of the junction based on 
  the parameters entered (left turn lanes, bus lanes etc). This makes it easier 
  for the user to understand how exactly their settings affect the design of a 
  junction and if they are what they intended.
  \item Priority: Should-have 
  \item Verification: The representation can be generated as an image and then 
  that image compared to a generated image that has been checked to be correct. 
  This would be able to be unit tested but should also be used  with functional 
  testing.
\end{itemize}

\section{System Evolution}
% Describes assumptions on which the system is based and anticipated changes due to changing user needs, and hardware evolution.

\section{Project Philosophy}

\subsection{Team Roles}
% playing to strengths blah blah

As a team we have been meeting once a week on Wednesdays and will continue to do 
so until the end of the project. When recording the Dragon's Den video presentation 
we will allocate some more time as well as making sure that someone who's familiar 
with video editing software is able to fully focus on the video to make it as good
as possible. Our team is comprised of the following members:

\begin{itemize}
  \item Krister
  \item Josh 
  \item Antoni
  \item Eshan
  \item Thomas 
  \item Ani
\end{itemize}

When selecting who will do each section we will play to the strength's of the 
team, for example Josh is good at backend so we'll focus his effort away from 
the front end and onto the simulation aspects. We have setup a discord channel which 
allows for quick communication between the team as well as voice and screen sharing 
to allow us to discuss things more effectively than just over text.

\subsection{Development Philosophy}
% waterfall/reuse driven development
We will utilise a hybrid approach combining the main themes of Waterfall with a 
reuse oriented methodology for the software development part of the project. We 
have strict deadlines for each part of the waterfall cycle. The following are 
those timelines, they are spaced to allow sufficient time for each section as 
well as allowing for the whole team to review each stage and make any corrections 
we deem necessary. 

\begin{itemize}
  \item Requirement Analysis
        - 22nd January
  \item Planning and Design
        - 29th January
  \item Implementation and testing
        - 19th February
  \item Dragon's Den Video and Final Report
        - 26th February
\end{itemize}


\end{document}
