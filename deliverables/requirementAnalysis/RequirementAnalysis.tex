\documentclass{article}
% set up the page formatting
\usepackage[a4paper, portrait, margin=2.5cm]{geometry}
\usepackage{multicol}
\usepackage{fancyhdr}
% editable bits
\title{CS261 Group 29 Requirement Analysis}
\author{Add your names here}
\date{January 2025}
\fancyfoot[L]{Requirement Anlaysis}
\fancyfoot[R]{\thepage}

\begin{document}
\maketitle

\section{Given Requirements for document}
Each team is required to provide a Requirements Analysis Report. This will 
document your teams understanding of the specification. As well as addressing 
the specification itself, the report should also document any decisions that 
have been made in terms of group management (i.e. how you have organized 
yourselves to meet the goals of this project, how often you will meet, how you 
will address decision making etc.) and provide a list of requirements that 
your team have decided will meet the customer's needs.
A requirements analysis report should provide an outline of the features that 
the final solution will provide. Remember that requirements need to be detailed, 
justified and verifiable. Consider your validation process carefully, along 
with why you've chosen to include certain features. This part of your first 
deliverable must not exceed four sides of A4. Any appendices or other attachments 
will be removed.

\section{Introduction}
% Justifies the need for the system and outlines what it will do.
\section{System Architecture}
% Presents a high-level overview of the system, showing distribution of functions across system modules.

\section{System Requirements Specification}
% Describes functional and non-functional requirements.
\subsection{Requirement 1}
\subsubsection{Customer Facing}
The user must be able to input the rate of traffic flow from 
each direction to each other direction via text fields (one per traffic flow) 
in the User Interface (UI).
\subsubsection{Developer Facing}
\begin{itemize}
  \item The system must accept input from the text fields, parse the input to 
  determine if each traffic flow is a valid number, and, if all traffic flows 
  are valid, allow the simulation to be ran.
  \item Priority: must
  \item Verification: running the simulation with one set of valid traffic 
  flow rates in the text fields, running it again with a different set in 
  the same fields and verifying the junction efficiency metrics have changed, 
  and then trying to run the simulation with an invalid value in at least one 
  text field (e.g. “TEST” for Eastbound Traffic Flow Exiting West)
  \item Traceability: Input Parameters section in the specification
\end{itemize}

\subsection{Requirement 2}
\subsubsection{Customer Facing}
The user must be able to adjust the following settings for each 
road entering the junction, affecting the results of the simulation: how many 
lanes there are (1 to 5), whether there is a left-turn lane, bus lane, cycle 
lane (all mutually exclusive, the latter two requiring separate traffic flow rates) 
or none of those three, whether there is a pedestrian crossing (requires the duration 
of time the crossing is active and the number of crossing requests per hour), 
and the traffic light sequencing priority (0 to 4, 0 meaning no priority, 4 
meaning highest priority).
\subsubsection{Developer Facing}
\begin{itemize}
  \item For each junction configuration, the system must take into account 
  its specific settings, adjusting the calculations performed by the simulation 
  for that specific configuration
  \item Priority: must
  \item Verification: run a simulation with default parameters (2 lanes per 
  road entering the junction, equal priority on all lights, and all other 
  settings set to off/No) as a reference, and then run a simulation per setting 
  with that setting adjusted, showing the results of each simulation is different 
  from the reference.
  \item Traceability: Configurable Parameters section in the specification
\end{itemize}

\subsection{Requirement 3}
\subsubsection{Customer Facing}
After running the simulation, for each junction configuration, 
the user must be provided with the following three junction efficiency metrics 
per road entering the junction, as well as an overall score generated from the 
three metrics for the whole configuration: average wait time, maximum wait time, 
and maximum queue length.
\subsubsection{Developer Facing}
\begin{itemize}
  \item When the simulation is ran, for each junction configuration, the system 
  must take the configuration and the input traffic flow rates (shared across all 
  configurations) and calculate the average wait time, maximum wait time and 
  maximum queue length per direction entering the junction, as well as combine 
  these metrics to calculate an overall score for the whole configuration
  \item Priority: must
  \item Verification: run the simulation once with two different configurations 
  or twice with a single configuration and different traffic flows, and verify 
  that there is a difference between the metrics and overall score of the 
  configurations/simulation runs
  \item Traceability: Output section in the specification
\end{itemize}

\subsection{Requirement 4}
\begin{itemize}
  \item The system must allow for comparison of one or more sets of input parameters against the metrics defined in section \ref{requirements_user}
\end{itemize}

\subsection{Non-functional Requirements}
Requirement 5
\begin{itemize}
  \item The system will show graphical representation of the junction based on the parameters entered (left turn lanes, bus lanes etc). This makes it easier for the user to understand how exactly their settings affect the design of a junction and if they are what they intended.
  \item Priority: Should-have 
  \item Verification: The representation can be generated as an image and then that image compared to a generated image that has been checked to be correct. This would be able to be unit tested but should also be used  with functional testing.
\end{itemize}

\section{Project Philosophy}
\subsection{Team Roles}
% playing to strengths blah blah

As a team we have been meeting once a week on Wednesdays and will continue to do 
so until the end of the project. When recording the Dragon's Den video presentation 
we will allocate some more time as well as making sure that someone who's familiar 
with video editing software is able to fully focus on the video to make it as good
as possible. Our team is comprised of the following members:

\begin{itemize}
  \item Krister - Backend 
  \item Josh - Frontend and Backend 
  \item Antoni - Backend 
  \item Eshan -  Frontend
  \item Thomas - Frontend
  \item Ani - Video, Frontend and Backend
\end{itemize}

As a team we have decided to forgo a project manager and opt for regular meetings 
with a shared understanding of the goals, if anyone has any concerns about the group 
structue or work distribution then this can be brought up at the whole group meeting 
to everyone, Everyone is an equal on the team.

\subsection{Development Philosophy}
% waterfall/reuse driven development
We will utilise a hybrid approach combining the main themes of Waterfall with a 
reuse oriented methodolgy for the software development part of the project. We 
have strict deadlines for each part of the waterfall cycle. The following are 
those timelines, they are spaced to allow sufficient time for each section as 
well as allowing for the whole team to review each stage and make any corrections 
we deem neccessary. 

\begin{itemize}
  \item Requirement Analysis
        - 22nd January
  \item Planning and Design
        - 29th January
  \item Implementation and testing
        - 19th February
  \item Dragon's Den Video and Final Report
        - 26th February
\end{itemize}

Between these stages we will be writing the deliverables as a team alongside,
during the weekly meetings we will check the progress of the various documents,
giving feedback about the changes and any things that should be altered to best 
fit the requirements and plan of the project.

\end{document}
