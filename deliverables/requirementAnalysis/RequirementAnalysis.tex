\documentclass{article}
% set up the page formatting
\usepackage[a4paper, portrait, margin=2.5cm]{geometry}
\usepackage{multicol}
\usepackage{fancyhdr}
% editable bits
\title{CS261 Group 29 Requirement Analysis}
\author{Add your names here}
\date{January 2025}
\fancyfoot[L]{Requirement Anlaysis}
\fancyfoot[R]{\thepage}

\begin{document}
\maketitle
\tableofcontents

\section{Given Requirements for document}
Each team is required to provide a Requirements Analysis Report. This will document your teams understanding of the specification. As well as addressing the specification itself, the report should also document any decisions that have been made in terms of group management (i.e. how you have organized yourselves to meet the goals of this project, how often you will meet, how you will address decision making etc.) and provide a list of requirements that your team have decided will meet the customer's needs.
A requirements analysis report should provide an outline of the features that the final solution will provide. Remember that requirements need to be detailed, justified and verifiable. Consider your validation process carefully, along with why you've chosen to include certain features.
This part of your first deliverable must not exceed four sides of A4. Any appendices or other attachments will be removed.

\section{Introduction}
% Justifies the need for the system and outlines what it will do.

\section{User Requirements Design}
% Describes the services provided for the users. Written in natural language with diagrams.

\section{System Architecture}
% Presents a high-level overview of the system, showing distribution of functions across system modules.

\section{System Requirements Specification}
% Describes functional and non-functional requirements.

\section{System Evolution}
% Describes assumptions on which the system is based and anticipated changes due to changing user needs, and hardware evolution.

\end{document}