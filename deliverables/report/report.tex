\documentclass{article}
% set up the page formatting
\usepackage[a4paper, portrait, margin=2.5cm]{geometry}
\usepackage{multicol}
\usepackage{fancyhdr}
\usepackage{graphicx}
\usepackage{float}

% allow for table of contents to have clickable links
\usepackage{hyperref}

% editable bits
\title{\vspace{-1.5cm}CS261 Group 29 Report}
\author{Ani Bitri, Krister Hughes, Thomas Phuong, Eshan Sharif, Josh Turner, Antoni Zyla}
\date{January 2025}
\fancyfoot[L]{Report}
\fancyfoot[R]{\thepage}

\begin{document}
\maketitle

\section{Preface}

Dorset Software tasked our team with creating a traffic junction simulator. And we fucking did it. 

Dorset Software has tasked our team with the creation of a traffic junction simulator which will be used for the modelling of traffic junctions on various parameters. The system provides data on how these configurations 
affect the traffic flow and the overall efficiency of the junction in comparsion to other configurations. 

This document explains the development process of the system and provide an explanation to the key changes and decisions made throughout the implementation of the prototype. Furthermore,
it will give an insight into the algorithms and formulas used to calculate the traffic flow and efficiency of the junctions.

\section{System Overview}

    \subsection{Developer Tools Used}

    \subsection{User Interaction}

    \subsection{System Architecture and Interaction}

\section{Modifications}

\section{Algorithms and Formulas}

    \subsection{Semaphores}

    \subsection{Traffic Flow}

    \subsection{Calculating Metrics}

    \subsection{Overall Score}

\section{Frontend}

    \subsection{Input and Validation}

    \subsection{Simulation and Visualisation Page}

    \subsection{Results Page}

    \subsection{Error Handling}

\section{Backend}

    \subsection{Data Classes}


\section{Testing}

    \subsection{Unit Testing}

    \subsection{User Testing}

    \subsection{Error Handling Testing}

\section{Product Evaluation}

\section{Process Evaluation}

\section{Conclusion}


\end{document}
